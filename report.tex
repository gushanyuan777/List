\documentclass[UTF8]{ctexart}
\usepackage{geometry, CJKutf8}
\geometry{margin=1.5cm, vmargin={0pt,1cm}}
\setlength{\topmargin}{-1cm}
\setlength{\paperheight}{29.7cm}
\setlength{\textheight}{25.3cm}

% useful packages.
\usepackage{amsfonts}
\usepackage{amsmath}
\usepackage{amssymb}
\usepackage{amsthm}
\usepackage{enumerate}
\usepackage{graphicx}
\usepackage{multicol}
\usepackage{fancyhdr}
\usepackage{layout}
\usepackage{listings}
\usepackage{float, caption}

\lstset{
    basicstyle=\ttfamily, basewidth=0.5em
}

% some common command
\newcommand{\dif}{\mathrm{d}}
\newcommand{\avg}[1]{\left\langle #1 \right\rangle}
\newcommand{\difFrac}[2]{\frac{\dif #1}{\dif #2}}
\newcommand{\pdfFrac}[2]{\frac{\partial #1}{\partial #2}}
\newcommand{\OFL}{\mathrm{OFL}}
\newcommand{\UFL}{\mathrm{UFL}}
\newcommand{\fl}{\mathrm{fl}}
\newcommand{\op}{\odot}
\newcommand{\Eabs}{E_{\mathrm{abs}}}
\newcommand{\Erel}{E_{\mathrm{rel}}}

\begin{document}

\pagestyle{fancy}
\fancyhead{}
\lhead{顾善元, 3230104463}
\chead{数据结构与算法第四次作业}
\rhead{Oct.20th, 2024}

\section{测试程序的设计思路}

我设计了一个简单的链表数据结构,并实现了插入和删除操作。首先创建了一个空链表,然后向其中插入了一系列元素,包括整数和字符串。接着,逐个删除这些元素,确保删除操作能够正确执行。我还测试了在空链表上执行删除操作的情况,以确保程序能够正确处理边界情况。  

除了基本的功能测试外,还对一些特殊情况进行了测试,比如在非法输入下程序的表现,以及对大量数据进行操作时的性能表现。确保程序在各种情况下都能正确运行。  

\section{测试的结果}

经过一系列测试,我确认程序在各种情况下都能正确执行。插入和删除操作都能如预期般运行,没有出现异常情况。针对边界情况的测试也表明程序能够正确处理空链表和其他特殊情况。  
  
此外,我使用了valgrind进行内存泄漏检测,结果显示程序没有发生内存泄漏问题。  
  


\end{document}

%%% Local Variables: 
%%% mode: latex
%%% TeX-master: t
%%% End: 
